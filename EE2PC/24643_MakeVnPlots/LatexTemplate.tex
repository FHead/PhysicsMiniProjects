% Set border=0 if you have found  the coordinate 
% at which you place the front image on the back image.
\documentclass[pstricks,border=0pt]{standalone}
\usepackage{pstricks}
\usepackage{graphicx}


% User defined data:
\def\BackImage{TempInput.pdf}
\def\FrontImage{MOD.pdf}

\def\ScaleBack{1}
\def\ScaleFront{0.2}

\def\Columns{10}
\def\Rows{10}

% Internal used data:
\newsavebox\IBack
\savebox\IBack{\includegraphics[scale=\ScaleBack,page=1]{\BackImage}}

\newsavebox\IFront
\savebox\IFront{\includegraphics[scale=\ScaleFront]{\FrontImage}}


\psset
{
    xunit=\dimexpr\wd\IBack/\Columns\relax,
    yunit=\dimexpr\ht\IBack/\Rows\relax,
}


\begin{document}

% Comment out the following code
% to locate the coordinate to overlay the front image on the back image.
%\begin{pspicture}(\Columns,\Rows)
%   \rput[bl](0,0){\usebox\IBack}
%   \psgrid
%\end{pspicture}


\newcount\x
\IfFileExists{\BackImage}{%
    \loop
            \ifnum\x<\XeTeXpdfpagecount"\BackImage"\relax
            \advance\x by 1\relax
            \savebox\IBack{\includegraphics[page=\the\x,scale=\ScaleBack]{\BackImage}}%
            \begin{pspicture}(\Columns,\Rows)
                        \rput[bl](0,0){\usebox\IBack}
                        \rput(0.875\wd\IBack,0.875\ht\IBack){\usebox\IFront}
            \end{pspicture}         
    \repeat
}{%
    \noindent\hfill NOTHING! \hfill\null
}%

\end{document}

